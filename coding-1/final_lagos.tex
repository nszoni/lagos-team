% Options for packages loaded elsewhere
\PassOptionsToPackage{unicode}{hyperref}
\PassOptionsToPackage{hyphens}{url}
%
\documentclass[
  9pt,
]{article}
\usepackage{amsmath,amssymb}
\usepackage{lmodern}
\usepackage{iftex}
\ifPDFTeX
  \usepackage[T1]{fontenc}
  \usepackage[utf8]{inputenc}
  \usepackage{textcomp} % provide euro and other symbols
\else % if luatex or xetex
  \usepackage{unicode-math}
  \defaultfontfeatures{Scale=MatchLowercase}
  \defaultfontfeatures[\rmfamily]{Ligatures=TeX,Scale=1}
\fi
% Use upquote if available, for straight quotes in verbatim environments
\IfFileExists{upquote.sty}{\usepackage{upquote}}{}
\IfFileExists{microtype.sty}{% use microtype if available
  \usepackage[]{microtype}
  \UseMicrotypeSet[protrusion]{basicmath} % disable protrusion for tt fonts
}{}
\makeatletter
\@ifundefined{KOMAClassName}{% if non-KOMA class
  \IfFileExists{parskip.sty}{%
    \usepackage{parskip}
  }{% else
    \setlength{\parindent}{0pt}
    \setlength{\parskip}{6pt plus 2pt minus 1pt}}
}{% if KOMA class
  \KOMAoptions{parskip=half}}
\makeatother
\usepackage{xcolor}
\IfFileExists{xurl.sty}{\usepackage{xurl}}{} % add URL line breaks if available
\IfFileExists{bookmark.sty}{\usepackage{bookmark}}{\usepackage{hyperref}}
\hypersetup{
  pdftitle={Report on Snickers and Coca-Cola Prices},
  pdfauthor={Lagos Team},
  hidelinks,
  pdfcreator={LaTeX via pandoc}}
\urlstyle{same} % disable monospaced font for URLs
\usepackage[margin=2cm]{geometry}
\usepackage{graphicx}
\makeatletter
\def\maxwidth{\ifdim\Gin@nat@width>\linewidth\linewidth\else\Gin@nat@width\fi}
\def\maxheight{\ifdim\Gin@nat@height>\textheight\textheight\else\Gin@nat@height\fi}
\makeatother
% Scale images if necessary, so that they will not overflow the page
% margins by default, and it is still possible to overwrite the defaults
% using explicit options in \includegraphics[width, height, ...]{}
\setkeys{Gin}{width=\maxwidth,height=\maxheight,keepaspectratio}
% Set default figure placement to htbp
\makeatletter
\def\fps@figure{htbp}
\makeatother
\setlength{\emergencystretch}{3em} % prevent overfull lines
\providecommand{\tightlist}{%
  \setlength{\itemsep}{0pt}\setlength{\parskip}{0pt}}
\setcounter{secnumdepth}{-\maxdimen} % remove section numbering
\usepackage{booktabs}
\usepackage{longtable}
\usepackage{array}
\usepackage{multirow}
\usepackage{wrapfig}
\usepackage{float}
\usepackage{colortbl}
\usepackage{pdflscape}
\usepackage{tabu}
\usepackage{threeparttable}
\usepackage{threeparttablex}
\usepackage[normalem]{ulem}
\usepackage{makecell}
\usepackage{xcolor}
\usepackage{siunitx}
\newcolumntype{d}{S[input-symbols = ()]}
\ifLuaTeX
  \usepackage{selnolig}  % disable illegal ligatures
\fi

\title{Report on Snickers and Coca-Cola Prices}
\author{Lagos Team}
\date{7 November 2021}

\begin{document}
\maketitle

\hypertarget{introduction}{%
\subsection{Introduction}\label{introduction}}

To identify price change tendencies in two conventional products:
Coca-Cola 500ml plastic bottle and Snickers 50gr, within the city of
Budapest, particularly in the 10th and 13th districts, we want to see
what effects are caused by allocation of the product from inner versus
outer city. Our data set and artifacts can be found on our
\href{https://github.com/nszoni/lagos-team/coding-1/}{Github repo}

\hypertarget{data-quality-and-cleaning}{%
\subsection{Data Quality and Cleaning}\label{data-quality-and-cleaning}}

Data collection was conducted over the weekend of Sept 25th-26h between
noon and early afternoon. The quailty of data was enured througout the
process and a criteria was established for smaple selection.

Our data cleaning process had four main phases. Firstly, we replaced
`N/A' string with the conventional NA. Secondly, we did a dimensionality
reduction by aggregating group types into big and small types. Thirdly,
we replaced boolean encoding with binary dummy values. Lastly, we casted
all the columns to match their data types specified in the variables
table above and shortened the name of the products to reduce clutter on
charts and tables. ``ectreme valuess''

\hypertarget{descriptive-statistics}{%
\subsection{Descriptive Statistics}\label{descriptive-statistics}}

We are able to detect extreme values in our data set, which are items
that are found within gas stations, which could potentially tell us that
price rises based on momentum and desire, not strictly adhere to
location of the district, but an amplification of sampling would be
needed.

Price for snickers is almost standardized in both districts, since the
majority of data is concentrated in a certain price range, while for
coke the range is wider, which makes us look into the distance as
factor, for coke it seems that it does make an increase but not for
snickers.

\begin{table}[!h]

\caption{\label{tab:unnamed-chunk-3}Descriptive statistics of product prices by district}
\centering
\fontsize{8}{10}\selectfont
\begin{tabular}[t]{llrrrrrrrrrr}
\toprule
District &   & N & Missing & Mean & Median & SD & Min & Max & Range & P05 & P95\\
\midrule
10 & Coke 0.5l & 15 & \num{0} & \num{304.5} & \num{299} & \num{52.43} & 230 & 449 & \num{219} & \num{244} & \num{380}\\
 & Snickers 50g & 14 & \num{1} & \num{193.8} & \num{189} & \num{49.50} & 119 & 319 & \num{200} & \num{120} & \num{274}\\
13 & Coke 0.5l & 14 & \num{1} & \num{318.4} & \num{304} & \num{52.63} & 259 & 439 & \num{180} & \num{259} & \num{426}\\
 & Snickers 50g & 15 & \num{0} & \num{192.6} & \num{180} & \num{46.96} & 119 & 299 & \num{180} & \num{147} & \num{299}\\
\bottomrule
\multicolumn{12}{l}{\rule{0pt}{1em}}\\
\end{tabular}
\end{table}

Intuitively the mean of both products are slightly higher in the inner
13th district. Also there is less variation in price of Snicker bars and
more variation in the price of Coke. Those who are living in the inner
city have to pay a larger entry price for Coke. The distribution of
prices in teh inner district is shofted a bit considering the
percentiles.

\hypertarget{price-distribution}{%
\subsection{Price Distribution}\label{price-distribution}}

The team decide to select a comibination of both histogram and density
plot to show the distribution of prices for both the products in two
districts. Histogram with a bin size of 10 was used becuase the price
difference and frequecy of each price could be easily displayed, while
desity plot shows how the distribution looks like, making it easier to
visualise.

\begin{center}\includegraphics{final_lagos_files/figure-latex/unnamed-chunk-4-1} \end{center}

\hypertarget{price-and-distance}{%
\subsection{Price and Distance}\label{price-and-distance}}

To display the results of our second analysis, ``relationship between
price and distance from centre'' a scator plots were used to display for
both the product. For both products in District 13 the price increased
as you moved away from the city centre, while in District 10 price
decreased.

\begin{center}\includegraphics{final_lagos_files/figure-latex/unnamed-chunk-6-1} \end{center}

\end{document}
